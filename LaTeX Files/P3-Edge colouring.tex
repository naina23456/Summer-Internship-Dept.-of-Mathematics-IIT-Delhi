\documentclass[12pt]{beamer}
\usepackage[utf8]{inputenc}
\usepackage{tabularx}
\usepackage{hyperref}
%\usepackage{mathrsfs}
\usetheme{Copenhagen}

\usecolortheme{wolverine}

\title{Edge Coloring}
\author{Naina Kumari}
\institute{\begin{large}
IIT DELHI
\begin{figure}[h]
\includegraphics[scale=0.5]{logo_d}
\end{figure}

\end{large}}

\begin{document}
\maketitle

\begin{frame}
\begin{block}{What is edge coloring ? }
An edge coloring of a graph G is a mapping f :E(G) $\rightarrow$ S.The elements of S are colors . The edges colored by same color foem a color class . If $|S| = k $, then f is a k-edge coloring of G.
\end{block}
\end{frame}

\begin{frame}{Line Graph}
A line graph L(G) of a simple graph G is obtained by associating a vertex with each edge of the graph and connecting two vertices with an edge iff the corresponding edges of G have a vertex in common
\begin{figure}
\includegraphics[scale=0.2]{line_graph.png}
\end{figure}
\end{frame}

\begin{frame}{Important Points}
\begin{itemize}
\item Every edge coloring problem can be converted into a vertex coloring problem.
\item Coloring the edges of a graph G is same as coloring the vertices of the line graph L(G)
\end{itemize}
\end{frame}

\begin{frame}
In general , $\chi(G) \geq \Delta$ where $\Delta$ = max. degree of a vertex in G for any graph G .
\end{frame}

\begin{frame}
\begin{block}{Bipartite Graphs}
A bipartite graph is a graph in which the vertex set can be partitioned into two sets such that edges only go between sets, not within them.
\end{block}
\begin{figure}
\includegraphics[scale=0.7]{bip_graph.png}
\caption{Fig 5 : Bipartite Graph}
\end{figure}
\end{frame}



\begin{frame}{Bipartite Graphs}
\begin{block}{Konig's Theorem}
If G is bipartite , then $\chi(G) = \Delta$ where $\Delta$ = max. degree of a vertex in G
\end{block}

\begin{figure}
\includegraphics[scale=0.3]{bip.jpg}
\end{figure}
\end{frame}

\begin{frame}
Let $m = $ no. of edges in G \\
\underline{Base Case}: m $=$ 1\\
It is trivially true \\~\\

So , let's assume the statement is true for any graph G with m edges\\~\\
\underline{Inductive Step}: Let there be m+1 edges in a graph G with max. degree = $\Delta$ \\~\\
Claim: $\chi(G) = \Delta$
\end{frame}

\begin{frame}{Proof}
Remove an arbitrary edge say (A,B) from G , call this subgraph G'\\~\\
By Induction Hypothesis , G' can be colored with $\Delta$ colours .

\begin{figure}
\includegraphics[scale=0.5]{G'.png}
\end{figure}
\end{frame}

\begin{frame}

Since max. degree = $\Delta$, deg(A) $\leq \Delta - 1$\\
and deg(B) $\leq \Delta - 1$ \underline{in G'} \\
It means A and B can also use $\Delta$ colors each and so , both A and B each have one unused color , $\alpha$ and $\beta$ respectively.\\~\\

\begin{figure}
\includegraphics[scale=0.5]{case1.png}
\end{figure}

Case 1 : $\alpha = \beta$ ,\\

then we can use this colour to colour (A,B)

\begin{figure}
\includegraphics[scale=0.5]{case11.png}
\end{figure}
\end{frame}

\begin{frame}{}

Case 2 : $\alpha \neq \beta$ \\
\begin{figure}
\includegraphics[scale=0.4]{bip_1}
\end{figure}
Clearly , there is a blue edge incident on A (otherwise it would be case 1) , let that edge be incident on some vertex B1.

\begin{figure}
\includegraphics[scale=0.4]{bip_2}
\end{figure}

\end{frame}

\begin{frame}
Subcase 2.1 : There is no red edge incident on B1 . It means Red is available at B1 . \\
So , color (A,B1) with red .
\begin{figure}
\includegraphics[scale=0.4]{bip_3}
\end{figure}

Subcase 2.2 : There is a red edge incident on B1 .
\begin{figure}
\includegraphics[scale=0.4]{bip_4}
\end{figure}
\end{frame}

\begin{frame}
Subcase 2.2.1 : There is no blue edge incident on A1 .
\begin{figure}
\includegraphics[scale=0.4]{bip_7}
\end{figure}


\end{frame}

\begin{frame}
Subcase 2.2.2 : There is a blue edge incident on A1 . 
\begin{figure}
\includegraphics[scale=0.4]{bip_5}
\end{figure}

\end{frame}

\begin{frame}
\begin{figure}
\includegraphics[scale=0.4]{bip_8}
\end{figure}
\end{frame}

\begin{frame}
\begin{figure}
\includegraphics[scale=0.4]{bip_9}
\end{figure}
So , (A,B) gets coloured without using any extra color .
Therefore $\chi(G) = \Delta$ 
\end{frame}


\begin{frame}
Thus , the proof of Konig's Theorem is complete .
\end{frame}

\begin{frame}{References}
        \begin{itemize}
            \item Graph Theory with Applications to Engineering and Computer Science by Narsingha Deo
            \item GRAPH THEORY WITH APPLICATIONS by J. A. Bondy and U. S. R. Murty
        \end{itemize}
    \end{frame}

\begin{frame}
\LARGE{\textbf{THANK YOU}} 
\end{frame}

\end{document}