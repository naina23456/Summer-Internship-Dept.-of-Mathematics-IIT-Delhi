\documentclass[12pt]{beamer}
\usepackage[utf8]{inputenc}
\usepackage{tabularx}
\usepackage{hyperref}
%\usepackage{mathrsfs}
\usetheme{Copenhagen}

\usecolortheme{wolverine}

\title{Colourability of planar graphs}
\author{Naina Kumari}
\institute{\begin{large}
IIT DELHI
\begin{figure}[h]
\includegraphics[scale=0.5]{logo_d}
\end{figure}

\end{large}}
\begin{document}
\setbeamertemplate{caption}{\raggedright\insertcaption\par}
\maketitle

\begin{frame}{Contents}
\begin{itemize}
\item What is a Graph ?
\item Types Of Graphs 
\item Vertex Colouring
\item Chromatic Number 
\item Six Colour Theorem 
\item Five Colour Theorem 
\end{itemize}
\end{frame}

\begin{frame}{What is a Graph ?}
A graph (denoted as $G=(V,E))$ consists of a non-empty set of vertices or nodes $V$ and a set of edges $E$ \\
\end{frame}

\begin{frame}
\begin{example}
Let us consider, a Graph $G=(V,E)$ where $V={a,b,c,d}$ and $E={{a,b},{a,c},{b,c},{c,d}}$
\end{example}

\begin{figure}
\includegraphics[scale=0.8]{graph_def.jpg}
\caption{Fig 1 : Graph }
\end{figure}
\end{frame}


\begin{frame}
\begin{block}{Empty/Null Graph}
A null graph is a graph in which there are no edges between its \hyperlink{page.22}{vertices}. 
\begin{figure}
\includegraphics[scale=0.8]{null_graph.png}
\caption{Fig 2 : Empty Graph}
\end{figure}
\end{block}
\end{frame}

\begin{frame}
\begin{block}{Simple Graph}
A graph is simple if it has no loops and no two of its links join the same pair of vertices$($i.e. no multiple edges$)$.
A simple graph which has n vertices, the degree of every vertex is at most n -1.
\end{block}
\end{frame}

\begin{frame}
\begin{figure}
\includegraphics[scale=0.7]{simple_graph.png}
\caption{Fig 3}
\end{figure}

\end{frame}

\begin{frame}
\begin{block}{Connected Graphs}
A connected graph is a graph in which we can visit from any one vertex to any other vertex. In a connected graph, at least one edge or path exists between every pair of vertices.
\end{block}
\begin{figure}
\includegraphics[scale=0.7]{conn_graph.png}
\caption{Fig 4 : Connected Graph}
\end{figure}
\end{frame}

\begin{frame}
\begin{block}{Bipartite Graphs}
A bipartite graph is a graph in which the vertex set can be partitioned into two sets such that edges only go between sets, not within \hyperlink{page.22}{them} .
\begin{figure}
\includegraphics[scale=0.7]{bip_graph.png}
\caption{Fig 5 : Bipartite Graph}
\end{figure}

\end{block}
\end{frame}

\begin{frame}
\begin{block}{PLANAR GRAPHS}
A planar graph is a graph that we can draw in a plane in such a way that no two edges of it cross each other except at a vertex to which they are incident.

\end{block}
\begin{figure}
\includegraphics[scale=0.5]{planar-graph.jpg}
\caption{Fig 6 : Planar Graph}
\end{figure}
\end{frame}

\begin{frame}
\begin{block}{Euler's Formula for Planar Graphs}
For any connected planar graph with 
$v$ vertices, $e$ edges and $f$ faces, we have
$$ v-e+f=2 $$
\end{block}
\underline {It can be proved by induction .}
\end{frame}

\begin{frame}
\begin{block}{NON PLANAR GRAPHS}
A graph is said to be non planar if it cannot be drawn in a plane so that no edge cross.
\end{block}
\begin{figure}
\centering
\includegraphics[scale=0.5]{non_pl.jpg}
\caption{Fig 7 : Non Planar Graph}
\end{figure}
\end{frame}

\begin{frame}
\begin{block}{What is Vertex Colouring ?}
Vertex coloring is the procedure of assignment of colors to each vertex of a graph G such that no adjacent vertices get same color.
\end{block}
\begin{figure}
\centering
\includegraphics[scale=0.3]{v_c.png}
\caption{Fig 8 : Vertex Colouring}
\end{figure}
\end{frame}

\begin{frame}
\begin{block}{Remark}
The objective is to minimize the number of colors while coloring a graph.
\end{block}
\end{frame}



\begin{frame}
\begin{large}
We will restrict our discussion to simple graphs . Why ?
\end{large}
\end{frame}

\begin{frame}
\begin{block}{What is a subgraph ? }
A graph H is a subgraph of G (written H $\subset $ G) if $V(H) \subset V(G)$ and $ E(H)\subset E(G)$
\end{block}
\end{frame}

\begin{frame}
\begin{block}{What is a spanning subgraph ? }
A spanning subgraph (or spanning supergraph) of G is
a subgraph(or supergraph) H with $V(H) = V(G)$.
\end{block}
\begin{figure}
\includegraphics[scale=0.4]{spna.png}
\caption{Fig 9 : Spanning subgraph}
\end{figure}

\end{frame}

\begin{frame}
\begin{block}{Underlying Simple Graph}
By deleting from a graph G all loops and, for every pair of adjacent vertices , all but one link joining them, we obtain a simple spanning subgraph of G, called the underlying simple graph of G
\end{block}
\end{frame}

\begin{frame}{A graph and its underlying simple graph}
\begin{figure}
\includegraphics[scale=0.4]{und_sim.png}
\caption{Fig 10 : Underlying Simple Graph}
\end{figure}
\end{frame}


\begin{frame}
\begin{block}{Remark}
A graph can be vertex coloured with k colours iff its underlying simple graph can be vertex coloured with k colours .
\end{block}
\end{frame}

\begin{frame}{Chromatic Number}
The chromatic number $\chi(G)$
is the minimum positive integer k for which G is k colourable .\\

if $\chi(G) = k $, G is said to be
k-chromatic.
\end{frame}

\begin{frame}{Some Important Observations}
\begin{itemize}
\item[1] $)$ A simple graph is 1-colourable if and only if it is \hyperlink{page.5}{empty}  .
\item[2] $)$ A simple graph is 2-colourable if and only if it is \hyperlink{page.9}{bipartite} 
\end{itemize}
\end{frame}

\begin{frame}{Six Colour Theorem}
\begin{large}
The vertices of every planar graph can be colored using 6 colors
in such a way that no pair of vertices connected by an edge share the same color . 
\end{large}
\begin{figure}
\includegraphics[scale=0.6]{six.png}
\caption{Fig 11}
\end{figure}
\end{frame}

\begin{frame}{Lemma 1.1}
\underline {\large{Statement}}  \\
In every planar graph , there exists a vertex of degree at most 5 .\\
\underline {\large{Proof}}\\
Let , if possible , $\exists$ a planar graph s.t. all n vertices have degree $\geq$ 6 i.e. $d(v_i) \geq 6$ \\
By First Theorem of \hyperlink{page.27}{Graph Theory} , $\sum_{i=1}^{n} d(v_i) = 2e$\\
and $$e \leq 3v-6 $$ \\
$$\implies  2e \leq 6v-12$$  
\end{frame}

\begin{frame}
Also , we assumed $d(v_i) \geq 6$ , so \\
$$ 2e =  \sum_{i=1}^{n} d(v_i) \geq 6v  > 6v - 12$$
$$\implies 2e \geq 6v-12 $$ \\
which is contradiction to  $$e \leq 3v-6 $$
\end{frame}

\begin{frame}{Proof by induction}
If no. of vertices $($v$)\leq 6$ , then it is trivially true . \\
\begin{figure}
\includegraphics[scale=0.3]{five.png}
\caption{Fig 12}
\end{figure}
\end{frame}

\begin{frame}
\underline {\large{Base Case}} n = 1 is trivial .

Let us assume it is true for n = k vertices .\\
\underline {\large{Inductive Step}} : For n = k+1 \\
From \hyperlink{page.24}{Lemma 1.1} , G must contain a vertex of degree $\leq$ 5 . \\
Choose that vertex , denote it as u and now consider $G \setminus \{u\}$ , say G'\\
\begin{figure}
\includegraphics[scale=0.5]{6color2.png}
\caption{Fig 13}
\end{figure}
\end{frame}

\begin{frame}
Now ,clearly , a planar graph remains planar after removing a finite number of vertices , so \\
 G' is planar with k vertices . Now , we can use our induction hypothesis to say that G' can be coloured with 6 colours .\\
\end{frame}

\begin{frame}
Now we can think of this as colouring all of G except u . \\
But since , u has degree $\leq$ 5 , one of the six colours will not be used for any of the neighbours of u . 
Choose that colour to colour u . \\
Thus , G is coloured with 6 colours . \\
And so it is true for n = k+1 .\\
Since k was arbitrary , it is true for all natural numbers . 
\end{frame}

\begin{frame}{Five Colour Theorem}
\begin{large}
The vertices of every planar graph can be colored using 5 colors
in such a way that no pair of vertices connected by an edge share the same color . 
\end{large}
\end{frame}

\begin{frame}{Proof by induction}
If no. of vertices $($v$) \leq 5$ , then it is trivially true . \\
\underline {\large{Base Case}} n = 1 is trivial .

Let us assume it is true for n = k vertices .\\
\underline {\large{Inductive Step}} : For n = k+1 \\
From Lemma 1.1 , G must contain a vertex of degree $\leq$ 5 . \\
Choose that vertex , denote it as v and now consider $G \setminus \{v\}$ , say G'\\

Now ,clearly , a planar graph remains planar after removing a finite number of vertices , so \\
 G' is planar with k vertices . Now , we can use our induction hypothesis to say that G' can be coloured with 5 colours .\\
\end{frame}


\begin{frame}
\underline{Let's consider colouring of v} \newline \newline
\underline{\large{Case 1:}} deg(v) $\leq$ 4 , then it is trivial .\\
\begin{figure}
\includegraphics[scale=0.5]{five_11n.png}
\caption{Fig 14}
\end{figure}
\end{frame}

\begin{frame}
\underline{\large{Case 2:}} deg(v) = 5 , then there are two subcases .\newline \newline
\begin{figure}
\includegraphics[scale=0.5]{five_12n.png}
\caption{Fig 15}
\end{figure}
\end{frame}

\begin{frame}
Case 1 : $\exists$ no two vertices that are directly connected by an edge.
\begin{figure}
\includegraphics[scale=0.5]{five_13n.png}
\caption{Fig 15}
\end{figure}
\end{frame}

\begin{frame}
Case 2 : $\exists$ two vertices that are directly connected by an edge.
\begin{figure}
\includegraphics[scale=0.5]{five_14n.png}
\end{figure}
Suppose there is a path between vertices a and c, say a$\rightarrow v_1
\rightarrow v_2 \rightarrow ... \rightarrow v_n \rightarrow$ c.
\end{frame}




\begin{frame}
\begin{figure}
\includegraphics[scale=0.5]{five_15n.png}
\end{figure}
Color alternatively a by red, v1 by yellow, v2 by red, and so on.
\end{frame}

\begin{frame}
\begin{figure}
\includegraphics[scale=0.5]{five_15n.png}
\end{figure}
\end{frame}

\begin{frame}
According to the property of planar graphs, there is no similar path between b and d (because it will intersect the path a and c). Hence there is no path of alternating colors p2 and p4 through vertices b and d. So, vertex d can be painted with the color p2 and vertex b is still with the color p2. Now we have color p4 left over with which we can paint vertex v.\newline

\Large{Hence , the five colour theorem is proved .}
\end{frame}




\end{document}